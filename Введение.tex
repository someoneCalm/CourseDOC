\section*{ВВЕДЕНИЕ}
\addcontentsline{toc}{section}{ВВЕДЕНИЕ}

В современном мире аудиозаписи играют значительную роль в нашей повседневной жизни. От музыки, которую мы слушаем на наших устройствах, до голоса, который мы слышим на радио и телевидении, аудио является неотъемлемой частью нашего существования. В связи с этим возникает необходимость в инструментах для обработки и редактирования аудиофайлов. Аудиоредакторы – это программы, которые позволяют выполнять различные операции со звуковыми данными, такими как запись, редактирование, микширование, мастеринг и другие. Они предоставляют пользователям возможность работать со звуком на профессиональном уровне и создавать высококачественные аудиофайлы.

Актуальность аудиоредакторов обусловлена рядом причин. Во-первых, эти инструменты позволяют улучшить качество аудиозаписей, что важно для профессионального использования, например, в студиях звукозаписи или на радиостанциях. Во-вторых, они предоставляют возможность редактировать и микшировать аудиофайлы, что позволяет создавать уникальные звуковые эффекты и композиции. В-третьих, технологии кодирования и обработки звука обеспечивают более эффективное хранение данных, что делает их предпочтительным выбором для многих пользователей.

Таким образом, аудиоредакторы являются актуальными инструментами для работы со звуком, которые предоставляют широкие возможности для его обработки и улучшения. Они используются в различных областях, таких как звукозапись, радиовещание, создание видеоигр и других. С развитием технологий и появлением новых алгоритмов кодирования звука эти инструменты становятся все более мощными и функциональными, что делает их еще более актуальными для современных пользователей.

\emph{Цель настоящей работы} – разработка аудиоредактора для преобразований WAV аудиофайлов. Для достижения поставленной цели необходимо решить \emph{следующие задачи:}
\begin{itemize}
\item провести анализ предметной области;
\item разработать и спроектировать концептуальную модель программы;
\item реализовать программу средствами языка Python.
\end{itemize}

\emph{Структура и объем работы.} Отчет состоит из введения, 4 разделов основной части, заключения, списка использованных источников, 1 приложения. 
%Текст выпускной квалификационной работы равен \formbytotal{page}{страниц}{е}{ам}{ам}.

\emph{Во введении} сформулирована цель работы, поставлены задачи разработки, описана структура работы, приведено краткое содержание каждого из разделов.

\emph{В первом разделе} на стадии описания технической характеристики предметной области приводится сбор информации о предметной области.

\emph{Во втором разделе} на стадии технического задания приводятся требования к разрабатываемой программе.

\emph{В третьем разделе} на стадии технического проектирования представлены проектные решения для программы.

\emph{В четвертом разделе} приводится список классов и их методов, использованных при разработке прогораммы, производится тестирование разработанной программы.

В заключении излагаются основные результаты работы, полученные в ходе разработки.

В приложении А представлен графический материал.
В приложении Б представлены фрагменты исходного кода. 
