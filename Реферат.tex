\abstract{РЕФЕРАТ}

Объем работы равен \formbytotal{lastpage}{страниц}{е}{ам}{ам}. Работа содержит \formbytotal{figurecnt}{иллюстраци}{ю}{и}{й}, \formbytotal{tablecnt}{таблиц}{у}{ы}{} и \arabic{bibcount} библиографических источников. Количество приложений – 1. Фрагменты исходного кода представлены в приложении А.

Перечень ключевых слов: система, программа, редактор, интерфейс, пользователь, звук, аудиофайл, Python.

Объектом разработки является программа, представляющая собой набор иструментов для совершения операций над WAV аудиофайлами.

В процессе создания программы были выделены основные сущности путем создания информационных блоков, использованы классы и методы модулей, обеспечивающие работу с сущностями предметной области, а также корректную работу программы.

При разработке программы использовался язык программирования Python.

\selectlanguage{english}
\abstract{ABSTRACT}
  
The volume of work is \formbytotal{lastpage}{page}{}{s}{s}. The work contains \formbytotal{figurecnt}{illustration}{}{s}{s}, \formbytotal{tablecnt}{table}{}{s}{s} and \arabic{bibcount} bibliographic sources. The number of applications is 1. The program code is presented in annex A.

List of keywords: system, programm, editor, interface, use, sound, audiofile, Python.

The object of development is a program that is a set of tools for performing operations on WAV audio files.

In the process of creating the program, the main entities were identified by creating information blocks, classes and methods of modules were used to ensure work with the entities of the subject area, as well as the correct operation of the program.

When developing the program, the Python programming language was used.
\selectlanguage{russian}
