\section*{ЗАКЛЮЧЕНИЕ}
\addcontentsline{toc}{section}{ЗАКЛЮЧЕНИЕ}

Современный мир музыки и звукозаписи не может обойтись без аудиоредакторов. Они стали незаменимыми инструментами в музыкальной индустрии, радиовещании, телевидении, видеопроизводстве, кино и других областях, где необходимо обрабатывать и улучшать звуковые файлы. Соответственно, профессиональные навыки работы с аудиоредакторами являются очень актуальными на современном рынке труда. Владение компьютерными программами для звукозаписи и редактирования аудиофайлов открывает широкие возможности для работы в медиа-индустрии и музыкальном бизнесе. Кроме этого, возможности аудиоредакторов легко доступны и для любителей – они могут использоваться для создания и обработки интересных и креативных музыкальных проектов.


Основные результаты работы:

\begin{enumerate}
\item Проведен анализ предметной области.
\item Разработана концептуальная модель программы.
\item Разработана архитектура программы.
\item Реализован пользовательский интерфейс.
\item Проведено системное тестирование.
\end{enumerate}

Все требования, объявленные в техническом задании, были полностью реализованы, все задачи, поставленные в начале разработки проекта, были также решены.

Готовый рабочий проект представлен программой расширения .exe.
